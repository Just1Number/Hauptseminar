% !TEX root = main.tex
\subsection{Ausgangssituation}
	Fragetypen\\
	Wie viele Fragen
\begin{itemize}
	\item Gegenüberstellung der Fragen im IT-Report 2020 \parencite{huebner2020}
	\item Wie viele Krankenhäuser gibt es insgesamt
	\item Wie viele Krankenhäuser wurden befragt? \parencite[31]{huebner2020}
	\begin{itemize}
		\item 608 IT-Verantwortliche
		\item 492 Deutschland
		\item 49 Österreich
		\item 67 Schweiz
		\item Rücklaufquoten:
		\begin{itemize}
			\item 25,2\% (DE), 18,8\% (AT), 31,8\% (CH)
		\end{itemize}
	\end{itemize}
\end{itemize}
\subsection{Methodik}
\begin{itemize}
	\item Einfache Vergleichsmethodik
	\item Für jede Frage gibt es einen Gewinner
	\begin{itemize}
		\item Wann ist jemand ein Gewinner
		\item bsp Papier einscanne: Geht das direkt in ne datenbank? Wahrschienlich nicht
	\end{itemize}
	\item Jeder Prozess wird einzeln betrachtet
	\item Für jede Frage wird entschieden was einen Digitalisierungsfortschritt darstellt
\end{itemize}

\subsection{Gegenüberstellung der Länder}
\begin{itemize}
	\item Gewinner vorstellen
	\item Gemeinsamkeiten und Unterschiede aufzeigen
\end{itemize}
\subsection{Defizite / Defizitbereiche im Vergleich}