% !TEX root = ../main.tex
\subsection{Ausgangssituation}
\begin{itemize}
	\item Fragetypen
	\item Wie viele Fragen
	\item Umfrage ist freiwillig
	\item 2020 kein WCS -> eigene Methodik für Vergleich
	\item Gegenüberstellung der Fragen im IT-Report 2020 \parencite{huebner2020}
	\item Wie viele Krankenhäuser gibt es insgesamt
	\item Wie viele Krankenhäuser wurden befragt? \parencite[31]{huebner2020}
	\begin{itemize}
		\item 608 IT-Verantwortliche
		\item 492 Deutschland
		\item 49 Österreich
		\item 67 Schweiz
		\item Rücklaufquoten:
		\begin{itemize}
			\item 25,2\% (DE), 18,8\% (AT), 31,8\% (CH)
		\end{itemize}
	\end{itemize}
\end{itemize}
\subsection{Methodik}
\begin{itemize}
	\item Ablesemethodik
	\begin{itemize}
		\item Bei Tortendiagrammen wurden Werte unter 10\% nicht dazu geschrieben, deshalb wurden sie für große n geschätzt. Für kleine Anzahl der Grundgesamtheit konnten die fehlenden Werte ermittelt werden, da nicht viele in Frage kamen.
	\end{itemize}
	\item Einfache Vergleichsmethodik
	\item Fokus: Universitätsmedizin wird ignoriert
	\item Für jede Frage gibt es einen Gewinner
	\item Wann ist jemand ein Gewinner?
	\item \textit{Weiß nicht} wird ignoriert
	\begin{itemize}
		\item wird indirekt neutral oder negativ bewertet
	\end{itemize}
	\item bsp Papier einscannen: Geht das direkt in ne datenbank? Wahrschienlich nicht
	\begin{itemize}
		\item Boxplots
		\item Boxplots erklären (https://hds.hebis.de/thm/Record/HEB363729879) S. 38
		\begin{itemize}
			\item Größter Median > größtes oberstes Quartil > größter oberster Whisker
			\item evtl. wie bei Güte der Informationsversorgung
		\end{itemize}
		\item Einfache Säulen mit wenig Antwortmöglichkeiten
		\begin{itemize}
			\item Summe der Antworten, die Digitalisierung anzeigen
			\item Größte Summe gewinnt
		\end{itemize}
		\item Ja/Nein Säulen mit vielen Kategorien
		\begin{itemize}
			\item Jede Kategorie gewinnt einzeln
			\item Land mit den meisten Kategorien gewinnt Frage
			\item Was passiert wenn Länder gleich viele Kategorien gewinnen
		\end{itemize}
		\item Säule mit mehr als zwei Teilen (Wie heißen diese Darstellungen richtig?)
		\item Güte der Informationsversorgung (Schweiz hat bei Aufnahme nicht teilgenommen)
		\begin{itemize}
			\item Antworten werden gewichtet
			\item ''Stimme überhaupt nicht zu'' = -2 bis ''Stimme voll zu'' = +2
			\item multipliziert mit Prozent der Stimmen und dann addiert
			\item Scores für alle Kategorien summieren
			\item höchster Score gewinnt
		\end{itemize}
	\end{itemize}
	\item Jeder Prozess wird einzeln betrachtet
	\item Für jede Frage wird entschieden was einen Digitalisierungsfortschritt darstellt
\end{itemize}

\subsection{Gegenüberstellung der Länder}
\begin{itemize}
	\item Gewinner vorstellen
	\item Gemeinsamkeiten und Unterschiede aufzeigen
\end{itemize}
\subsection{Andere Ländervergleiche}
\subsection{Defizite / Defizitbereiche im Vergleich}