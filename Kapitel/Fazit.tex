% !TEX root = ../main.tex
In dieser Arbeit wurde gezeigt, dass Deutschland im internationalen Vergleich mit Ländern ähnlicher Wirtschaftsstärke und Kultur beim Thema Digitalisierung im Krankenrausbereich hinten liegt. Dies zeigt sich z.B. beim unterdurchschnittlichen Einsatz von mobilen Geräten wie Tablets und bedside Terminals. Einen ähnlichen Rückschluss lassen die Bereiche Bildung, öffentliche Verwaltung oder Finanzdienstleistungen zu. Auch die unterdurchschnittliche Mobilfunk- und Breitbandabdeckung ist häufig Thema in den Medien.\\

Allgemein werden im Krankenhaus viele Informationen über Papier weitergegeben, vor allem an den externen Schnittstellen, wie Aufnahme und Entlassung. Das zeigt sich vor allem an der geringen elektronischen Vernetzung im gesamtheitlichen Gesundheitswesen. Es gibt die berechtigte Hoffnung, dass sich diese Situation durch eine flächendeckende und einheitliche Implementierung der elektronischen Patientenakte verbessert.\\

In Krankenhäusern wird das Prinzip der IT-Governance selten angewandt. D.h., der Einsatz von IT ist nicht Bestandteil der strategischen Ziele und somit bleibt die IT hinter ihren Möglichkeiten zur Unterstützung der Krankenhausabläufe zurück.\\

Die aktuell hohe Auslastung der Krankenhäuser durch die Pandemie könnte sich negativ auf den Digitalisierungsfortschritt auswirken: Überlastetes Personal ist schwer für die Einführung neuer Systeme zu gewinnen. Zwar führt Digitalisierung langfristig zur Vereinfachung und Optimierung von Prozessen, ist aber eben auch mit einer Einarbeitungs- und Umstellungsphase verbunden. Andererseits erfüllen Krankenhausbereiche, deren Kapazitäten gerade durch die Pandemie unausgelastet sind, evtl. die Vorraussetzungen, um aufgestaute IT-Projekte voranzutreiben. Nachdem sich die Situation in den Krankenhäusern entspannt hat, heißt es allerdings aus diesem Stresstest der Prozesse zu lernen und zu erarbeiten, wo der Einsatz von neuen Technologien sinnvoll die Arbeit der Kliniker ergänzt.\\

Zuversichtlich für die Zukunft stimmt in Punkto Digitalisierung das Eintreten der \textit{Digital Natives} ins Berufsleben. Dieser neuen Generation von Ärzten und Pflegern wird es grundsätzlich leichter fallen, neue Technik anzunehmen und einzusetzen. Erschwert wird die Digitalisierung allerdings durch den zunehmenden Fachkräftemangel auf Seiten der Pflegekräfte und IT-Spezialisten. Je ausgereifter die digitalen Lösungen sind und je mehr positive Beispiele es für die Implementierung gibt, desto leichter wird auch die Einführung dieser Technologien für die gesamte Krankenhauslandschaft sein. Krankenhäuser müssen hierzulande für solche positiven Beispiele tatsächlich nicht weit schauen, denn auch heute schon gibt es in Deutschland hochmoderne Krankenhäuser, wie z.B. das Universitätsklinikum Hamburg-Eppendorf \parencite{Baehr2019}. Insgesamt ist das Fortschreiten der Digitalisierung im Krankenhaus und im Gesundheitswesen alternativlos. Je entschlossener man sich dieser Herausforderung stellt, desto früher werden Abläufe verbessert und desto früher werden Patienten davon profitieren.
