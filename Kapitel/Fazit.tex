% !TEX root = ../main.tex
Ähnlich wie in vielen anderen Sektoren liegt Deutschland im internationalen Vergleich mit Ländern ähnlicher Wirtschaftsstärke und Kultur beim Thema Digitalisierung auch im Krankenhausbereich hinten. Dies zeigt sich z.B. beim unterdurchschnittlichen Einsatz von mobilen Geräten wie Tablets und bedside Terminals.\\

Allgemein werden viele Informationen über Papier weitergeben, vor allem bei externen Schnittstellen, wie der Aufnahme und Entlassung. Dies liegt vor allem an der geringen elektronischen Vernetzung im gesamtheitlichen Gesundheitswesen. Es gibt die Hoffnung, dass sich diese Situation durch eine flächendeckende und einheitliche Implementierung der elektronischen Patientenakte verbessert.\\

In Krankenhäusern wird das Prinzip der IT-Governance selten angewandt und somit ist der Einsatz von IT nicht an vorderster Stelle, sondern Mittel zum Zweck.\\

Die aktuell hohe Auslastung der Krankenhäuser durch die Pandemie könnte sich auch negativ auf den Digitalisierungsfortschritt auswirken, da überlastetes Personal schwer für die Einführung neuer Systeme zu gewinnen ist. Dies ist durchaus berechtigt, da eine solche Änderung zwar langfristig Prozesse optimieren kann, aber immer auch eine Einarbeitungsphase hat. Andererseits haben Bereiche, die von der Pandemie nicht direkt betroffen sind, eventuell mehr Kapazitäten um aufgestauten IT-Projekte voranzutreiben. Nachdem diese unmittelbaren Probleme gelöst werden und sich die Situation entspannt, heißt es allerdings aus diesem Stresstest der Prozesse zu lernen und zu erarbeiten, wo der Einsatz von neuen Technologien sinnvoll die Arbeit der Kliniker ergänzt. Dabei wird es der neuen Generation von Ärzten und Pflegern, die vermehrt aus \textit{digital Natives} besteht, grundsätzlich leichter fallen neue Technik anzunehmen.\\

Die Einführung dieser Technologien wird zudem immer leichter, da Systeme ausgereifter sind und es mehr Beispiele für deren erfolgreicher Implementierung gibt. Dabei müssen deutsche Krankenhäuser auch nicht weit schauen, da es auch heute schon hochmoderne Krankenhäuser in Deutschland gibt, wie z.B. das Universitätsklinikum Hamburg-Eppendorf \parencite{Baehr2019}.
