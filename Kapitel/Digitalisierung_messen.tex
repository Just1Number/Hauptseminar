% !TEX root = ../main.tex
\subsection{Workflow Composite Score (WCS)}
	% <fig>
	%\begin{figure}[h]
	%	\centering
	%	\includegraphics[width=0.8\textwidth]{Bilder/WCS_Hübner_2018_S22.png}
	%	\caption{Hierarchie des Workflow Composite Scores \parencite[22]{huebner2018}}
	%	\label{fig:WCS}
	%\end{figure}
	Diese Kennzahl bedient sich des \textit{Prinzip der klinischen Informationslogistik}. Bei diesem Prinzip \glqq{}[...]sollte ein [Krankenhausinformationssystem] immer die richtigen Patienteninformation zur richtigen Zeit am richtigen Ort in der richtigen Qualität für die richtigen Personen bereitstellen.\grqq\parencite[36]{huebner2019} Der Workflow Composite Score misst inwieweit dieser Standard erreicht ist. Hierbei werden über 100 Einzelindikatoren erfasst und, wenn möglich, in 5 klinische Prozesse (Aufnahme, Visite, OP-Vorbereitung, OP-Nachbereitung, Entlassung) eingeteilt. Bei Krankenhäusern, die nicht über einen OP verfügen, werden nur Aufnahme, Visite und Entlassung betrachtet. Innerhalb dieser Prozesse werden die Indikatoren noch 4 Deskriptoren, Daten und Informationen, Funktionen, Integration und Distribution, aufgeteilt. \parencite{huebner2019}
	\vspace{\parheadvspace}\\
	\textbf{Daten und Informationen}\\
	Bei Daten und Informationen geht es um die Vielfalt der elektronisch verfügbaren Patienteninformationen. Teil dieses Deskriptors ist z.B. die Frage ob Patientenstammdaten bei der Aufnahme schon in elektronischer Form vorhanden sind oder ob sie erst manuell aufgenommen werden müssen. 
	\vspace{\parheadvspace}\\
	\textbf{Funktion}\\
	Funktion erfasst welche IT-Anwendungen zur Verfügung stehen, z.B. ein System zur Belegesteuerung, aber auch das Vorhandensein von WLAN.
	\vspace{\parheadvspace}\\
	\textbf{Integration}\\
	Inwieweit diese Anwendungen miteinander verknüpft sind zeigt der Deskriptor Integration auf. Er gibt also den Stand des Krankenhausinformationssystems wieder.
	\vspace{\parheadvspace}\\
	\textbf{Distribution}\\
	Distribution sind die Mittel mit denen Informationen an ihre Nutzer gebracht werden. Können Pflegekräfte Patientendaten nur an einem PC einsehen oder gibt es Tablets, die einfacher mit auf ein Zimmer genommen werden können?

	TODO: SCORING
\subsection{Professionalism of Information Management Composite Score (PIMCS)}
	Das IT-Management ist entscheident für den Fortschritt der Digitalisierung in jedem Unternehmen, so auch im Krankenhaus. Wie gut es seine Rolle einnimmt und durchführt misst der PIMCS. Dabei wird das Informationsmanagement von drei Ebenen aus betrachtet: der strategischen, taktischen und operativen Ebene. \parencite{huebner2019}
	\vspace{\parheadvspace}\\
	\textbf{Strategische Ebene}\\
	Das strategische IT-Management umfasst einen längerfristigen Plan zur Entwicklung der Informationssysteme. Dabei ist es wichtig, dass häufig Evaluationen der existierenden IT-Landschaft durchgeführt werden, um die IT-Strategie anzupassen. Diese Evaluation soll ins Besondere die Nutzer der IT-Systeme einbeziehen. Das verfolgte Ideal ist hier das Prinzip der IT-Governance, die ein Unternehmen beschreibt, in dem die IT die strategische Planung des Gesamtunternehmens vorantreibt.
	\vspace{\parheadvspace}\\
	\textbf{Taktische Ebene}\\
	Die Umsetzung der IT-Strategie ist Teil des taktischen IT-Managements. Dieses beinhaltet die Implementierung neuer IT-Systeme und ähnliche Änderungen am operativen Geschäft. Dabei wird ein gutes Projektmanagement benötigt.
	\vspace{\parheadvspace}\\
	\textbf{Operative Ebene}\\
	Auf dieser Ebene wird erfasst wie das Tagesgeschäft durch das IT-Management unterstützt wird. Betrachtet wird hier auch die Qualität des Support und die Häufigkeit von Schulungen für Nutzer der IT-Systeme.

	Für jede dieser Ebenen wird ähnlich wie beim WCS ein Score ermittelt und dann eine Summe für ein Endscore gebildet. \parencite{huebner2019}
\subsection{Innovation Capability Composite Score (ICCS)}
	Eine wichtige Eigenschaft eines fortschrittlichen Unternehmens ist seine Innovationsfähigkeit. Um diese im Bereich der IT in Krankenhäusern zu messen wurde der ICCS entwickelt. Die IT wird zu diesem Zweck aus drei verschiedenen Blickwinkeln betrachtet, deren einzelne Bewertungen zum Schluss zu einem Gesamtscore zusammengezählt werden: Der IT-Innovationskultur, dem IT-Intrapreneurship der IT-Leitung und der Offenheit der IT gegenüber Anwendern. \parencite{huebner2019}
	\vspace{\parheadvspace}\\
	\textbf{IT-Innovationskultur}\\
	Um Mitarbeiter zu motivieren neue Ansätze vorzuschlagen und sich bei der Planung einzubringen, ist es nötig eine entsprechende Kultur zu schaffen. Dafür sind agile und flexible Organisationsstrukturen nötig. Eine Weiterentwicklung der IT muss durch die Krankenhausleitung unterstützt werden und eine klare IT-Zukunftsvision gibt Rückhalt.
	\vspace{\parheadvspace}\\
	\textbf{IT-Intrapreneurship der IT-Leitung}\\
	Das Vorantreiben des IT-Fortschritts ist Aufgabe der IT-Leitung. Eine starke intrinsische Motivation und progressive Denk- und Handelsweise sind hier nötig. Die IT-Leitung sollte versuchen festgefahrene Strukturen und Verhaltensweisen aufzuweichen und die Entstehung solcher verhindern.
	\vspace{\parheadvspace}\\
	\textbf{Offenheit gegenüber Anwendern}\\
	Die IT im Krankenhaus kann die klinischen Prozesse nur dann bestmöglich unterstützen, wenn das IT-Personal über hohe fachliche Kenntnisse verfügt. Neue IT-Anwendungen oder Änderungen zu existierender IT sind nur dann sinnvoll, wenn sie auf die tatsächlichen Abläufe zugeschnitten sind. Zu diesem Zweck ist eine enge Zusammenarbeit mit dem klinischen Endanwendern nötig.
\subsection{Electronic Medical Record Adoption Model (EMRAM)}
\label{sec:EMRAM}
	Das Electronic Medical Record Adoption Model (EMRAM) ist ein Modell, das einen Überblick über den Digitalisierungsgrad eines Krankenhauses aus Sicht der elektronisch verfügbaren Daten gibt. Es bedient sich eines acht-Stufen-Systems von der Stufe 0, welche ein Krankenhaus beschreibt, in dem kaum digital gearbeitet wird, bis zur Stufe 7, einem papierlosen Krankenhaus. Entwickelt wurde dieses Model von der \textit{HIMSS Analytics}, einem Tochterunternehmen der HIMSS (Healthcare Information and Management Systems Society), welches Krankenhäuser anbietet sich auf eine der Stufen zu zertifizieren. Diese benötigten Daten zur Zertifizierung erhält die HIMSS über einen Online-Fragebogen, den ein IT-Verantwortlicher des entsprechenden Krankenhauses ausfüllt. Bei einer Zertifizierung der Stufen 6 oder 7 wird das Krankenhaus zusätzlich durch einen Experten der HIMSS besucht.

	\parencite{Stephani2019}
