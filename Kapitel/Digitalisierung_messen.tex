% !TEX root = ../main.tex
\subsection{Workflow Composite Score (WCS)}
	\begin{figure}[ht]
		\centering
		\includegraphics[width=0.8\textwidth]{Bilder/WCS_Hübner_2018_S22.png}
		\caption{Hierarchie des Workflow Composite Scores \parencite[22]{huebner2018}}
		\label{fig:WCS}
	\end{figure}
	Der WCS ist eine Kennzahl, die sich des \textit{Prinzip der klinischen Informationslogistik} bedient. Bei diesem Prinzip \glqq{}[...]sollte ein [Krankenhausinformationssystem] immer die richtigen Patienteninformation zur richtigen Zeit am richtigen Ort in der richtigen Qualität für die richtigen Personen bereitstellen.\grqq\parencite[36]{huebner2019} Der Workflow Composite Score misst inwieweit dieser Anspruch erreicht ist. Hierbei werden über 100 Einzelindikatoren erfasst und, wenn möglich, in die fünf klinische Prozesse (Aufnahme, Visite, OP-Vorbereitung, OP-Nachbereitung, Entlassung) eingeteilt. Bei Krankenhäusern, die nicht über einen OP verfügen, werden nur Aufnahme, Visite und Entlassung betrachtet. Innerhalb dieser Prozesse werden die Indikatoren nach vier Deskriptoren (Daten und Informationen, Funktionen, Integration und Distribution) aufgeteilt. \parencite{huebner2019}
	\vspace{\parheadvspace}\\
	\textbf{Daten und Informationen}\\
	Bei Daten und Informationen geht es um die Vielfalt der elektronisch verfügbaren Patienteninformationen. Teil dieses Deskriptors ist z.B. die Frage, ob Patientenstammdaten bei der Aufnahme schon in elektronischer Form vorhanden sind oder ob sie erst manuell aufgenommen werden müssen. 
	\vspace{\parheadvspace}\\
	\textbf{Funktion}\\
	Funktion erfasst, welche IT-Anwendungen zur Verfügung stehen, z.B. ein System zur Belegungssteuerung, aber auch das Vorhandensein von WLAN.
	\vspace{\parheadvspace}\\
	\textbf{Integration}\\
	Inwieweit diese Anwendungen miteinander verknüpft sind, zeigt der Deskriptor Integration auf. Er gibt also den Zustand des Krankenhausinformationssystems wieder.
	\vspace{\parheadvspace}\\
	\textbf{Distribution}\\
	Distribution beschreibt die Mittel mit denen Informationen an ihre Anwender gebracht werden. Können Pflegekräfte Patientendaten nur an einem PC einsehen oder gibt es Tablets, die einfacher mit auf ein Zimmer genommen werden können?\\

	Für jeden Deskriptor in jedem Prozess wird ein Score ermittelt. Die Summe der Scores eines Prozesses über alle Deskriptoren ergibt den Prozess-(Sub)Score. Die Summe der Scores eines Deskiptoren über alle Prozesse ergibt den Deskriptoren-(Sub)score. Aus den Subscores ergibt sich ein Gesamtscore zwischen 1 und 100.(Siehe Abbildung \ref{fig:WCS})
\subsection{Professionalism of Information Management Composite Score (PIMCS)}
	Das IT-Management ist entscheidend für den Fortschritt der Digitalisierung in jedem Unternehmen, so auch im Krankenhaus. Wie gut es seine Rolle erfüllt und durchführt, misst der PIMCS. Dabei wird das Informationsmanagement von drei Ebenen aus betrachtet: der strategischen, taktischen und operativen Ebene. \parencite{huebner2019}
	\vspace{\parheadvspace}\\
	\textbf{Strategische Ebene}\\
	Das strategische IT-Management umfasst einen längerfristigen Plan zur Entwicklung der Informationssysteme. Dabei ist es wichtig, dass häufig Evaluationen der existierenden IT-Landschaft durchgeführt werden, um iterativ die IT-Strategie anzupassen. Diese Befragung soll insbesondere die Nutzer der IT-Systeme einbeziehen. Das verfolgte Ideal ist hier die so genannten IT-Governance, die eine Organisationsform in Unternehmen beschreibt, in dem die IT die strategische Planung des Gesamtunternehmens vorantreibt.
	\vspace{\parheadvspace}\\
	\textbf{Taktische Ebene}\\
	Die Umsetzung der IT-Strategie ist Teil des taktischen IT-Managements. Dieses beinhaltet die Implementierung neuer IT-Systeme und Änderungen am operativen Geschäft. Dabei wird ein gutes Projektmanagement benötigt.
	\vspace{\parheadvspace}\\
	\textbf{Operative Ebene}\\
	Auf dieser Ebene wird erfasst wie das Tagesgeschäft durch das IT-Management unterstützt wird. Betrachtet wird hier auch die Qualität des Supports und die Häufigkeit von Qualitfizierung für Nutzer der IT-Systeme.\\

	Für jede dieser Ebenen wird ähnlich wie beim WCS ein Score ermittelt und dann eine Summe für einen Endscore gebildet. \parencite{huebner2019}
\subsection{Innovation Capability Composite Score (ICCS)}
	Eine wichtige Eigenschaft eines erfolgreichen Unternehmens ist seine Innovationsfähigkeit. Um diese im Bereich der IT in Krankenhäusern zu messen, wurde der ICCS entwickelt. Die IT wird zu diesem Zweck aus drei verschiedenen Blickwinkeln betrachtet, deren einzelne Bewertungen zum Schluss zu einem Gesamtscore zusammengezählt werden: der IT-Innovationskultur, dem IT-Intrapreneurship der IT-Leitung und der Offenheit der IT gegenüber Anwendern. \parencite{huebner2019}
	\vspace{\parheadvspace}\\
	\textbf{IT-Innovationskultur}\\
	Um Mitarbeiter zu motivieren, neue Ansätze zu entwickeln und sich bei der Planung einzubringen, ist es nötig, eine entsprechende Kultur zu schaffen. Dafür sind agile und mitarbeiterzentrierte Organisationsstrukturen nötig. Die Weiterentwicklung der IT muss durch die Krankenhausleitung unterstützt werden und eine klare IT-Zukunftsvision gibt Orientierung.
	\vspace{\parheadvspace}\\
	\textbf{IT-Intrapreneurship der IT-Leitung}\\
	Das Vorantreiben des IT-Fortschritts ist Aufgabe der IT-Leitung. Eine starke intrinsische Motivation und unternehmerische Denk- und Handelsweise sind hier erfolgskritisch. Die IT-Leitung sollte versuchen, festgefahrene Strukturen und Verhaltensweisen aufzuweichen oder gar die Entstehung solcher verhindern.
	\vspace{\parheadvspace}\\
	\textbf{Offenheit gegenüber Anwendern}\\
	Die IT im Krankenhaus kann die klinischen Prozesse nur dann bestmöglich unterstützen, wenn der IT-Bereich über hohe Fach- und Prozesskenntnisse verfügt. Neue IT-Anwendungen oder Änderungen an existierender IT sind nur dann sinnvoll, wenn sie auf die realen Abläufe zugeschnitten sind (Anwenderorientierung). Zu diesem Zweck ist eine enge Zusammenarbeit zwischen dem IT-Entwickler und dem klinischen Endanwendern nötig.
\subsection{Electronic Medical Record Adoption Model (EMRAM)}
\label{sec:EMRAM}
	Das Electronic Medical Record Adoption Model (EMRAM) ist ein Modell, das einen Überblick über den Digitalisierungsgrad eines Krankenhauses aus Sicht der elektronisch verfügbaren Daten gibt. Es bedient sich eines acht-Stufen-Systems von der Stufe 0, welche ein Krankenhaus beschreibt, in dem kaum digital gearbeitet wird, bis zur Stufe 7, einem papierlosen Krankenhaus. Entwickelt wurde dieses Model von der \textit{HIMSS Analytics}, einem Tochterunternehmen der HIMSS (Healthcare Information and Management Systems Society), welches Krankenhäusern anbietet sich auf eine der Stufen zu zertifizieren. Diese benötigten Daten zur Zertifizierung erhält die HIMSS über einen Online-Fragebogen, den ein IT-Verantwortlicher des entsprechenden Krankenhauses ausfüllt. Bei einer Zertifizierung der Stufen 6 oder 7 wird das Krankenhaus zusätzlich durch einen Experten der HIMSS besucht.

	\parencite{Stephani2019}
