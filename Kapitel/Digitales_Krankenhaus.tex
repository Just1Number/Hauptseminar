% !TEX root = ../main.tex
\subsection{Chancen und Risiken}
Digitalisierung birgt viele Chancen der Verbesserung der Abläufe in einem Krankenhaus. Diese reichen von der Organisation bis hin zu medizinisch fachlichen Innovationen (Siehe Tabelle \ref{tab:hoffnung_risiken}). So wird eine Erleichterung der Dokumentation und eine Verbesserung der Kommunikation versprochen. Das Personal soll bei körperlich anstrengenden Tätigkeiten und Routineaufgaben entlastet werden. Die dadurch gewonnene Zeit wird bestenfalls für den direkten Patientenkontakt genutzt. Durch neue Technologien wie Operationsroboter kann die Diagnostik und Behandlung direkt verbessert oder in manchen Fällen erst möglich gemacht werden. Außerdem können sich Patienten evtl. den Weg zum Krankenhaus komplett sparen, wenn sie aus der Ferne betreut werden können. \parencite{braeutigam2017}

Gleichzeitig gibt es, wie bei jedem Fortschritt, Risiken, die es zu beachten gilt. So kann eine zunehmend digitale Datenlagerung die Datensicherheit reduzieren. Grade im medizinischen Bereich ist diese von enormer Wichtigkeit. Es gibt beim Personal die Befürchtung, das eventuelle Zeitersparnis nicht dem Patienten zu gute kommt sondern zu einer Reduzierung von Arbeitsplätzen führt. Die neuen Aufgaben, die das Personal bewältigen muss, sind komplex und durch intensivere Dokumentation und daraus folgende Überwachung kann der Leistungsdruck steigen. Neue medizintechnische Technologien erfordern neue Kompetenzen und neue Systeme sind störanfällig. \parencite{braeutigam2017}
\begin{table}[h]
	\begin{tabular}{l|p{.4\textwidth}|p{.4\textwidth}}
	\textbf{Bereiche}&\textbf{Chancen}&\textbf{Risiken}\\
	\hline
	\multirow{3}{*}{Organisation}
		&$\bullet$ die Erleichterung der Dokumentation &$\bullet$ mangelnde Datensicherheit\\
		&$\bullet$ Verbesserung bei Organisation und Kommunikation &$\bullet$ fehlende Akzeptanz der Beschäftigten\\
	\hline
	\multirow{7}{*}{Personal}
		&$\bullet$ Entlastung von körperlich anstrengenden Tätigkeiten &$\bullet$ wachsender Leistungsdruck und zusätzliche Aufgaben\\
		&$\bullet$ Entlastung von Routineaufgaben &$\bullet$ Überforderung der Beschäftigten\\
		&$\bullet$ Zeitersparnis&$\bullet$ zunehmende Kontrolle und Fremdbestimmung bei der Arbeit\\
		&&$\bullet$ Substitution von Tätigkeiten und Arbeitsplatzverluste\\
	\hline
	\multirow{6}{*}{Fachlich}
		&$\bullet$ Fernbetreuung von Patienten&$\bullet$ Störanfälligkeit\\
		&$\bullet$ die Verbesserung der Versorgungsqualität &$\bullet$ mangelnde Kompetenz bei Beschäftigten\\
		&$\bullet$ Qualitätssteigerungen bei Diagnostik und Therapie, insbesondere bei Operationen&\\
		&$\bullet$ mehr Zeit für den direkten Patientenkontakt vor allem in der Pflege&\\
	\end{tabular}
	\caption{Zusammenstellung der Hoffnungen und Risiken der Digitalisierung an Krankenhäusern \parencite{braeutigam2017}}
	\label{tab:hoffnung_risiken}
\end{table}
\subsection{Behandlungstechnologien}
	Das Kerngeschäft eines Krankenhauses ist die Behandlung von Patienten. Es gibt einige Technologien, mit denen der Patient direkt in Kontakt kommt.
	\cite{braeutigam2017} nennt hier Telemonitoring, mHealth, Wearable Computing, Operations- und Pflege-Roboter.
	\vspace{\parheadvspace}\\
	\textbf{Telemonitoring}\\
	Durch Telemonitoring kann sich ein Patient den Weg zum Krankenhaus sparen kann, indem er bestimmte Messungen selbst durchführt. So kann z.B. ein Blutdruckpatient seinen eigenen Blutdruck messen und über eine IT-Anwendung digital an den Arzt übermitteln. 
	\vspace{\parheadvspace}\\
	\textbf{mHealth}\\
	Eng mit Telemonitoring verbunden ist mHealth. mHealth umfasst Anwendungen auf mobilen Geräten, wie Smartphones, die bei der Behandlung unterstützen. Dabei geht es um das übermitteln von Daten, wie etwa Blutdruck, an einen Arzt aber auch das direkte Messen von gesundheitsrelevanten Daten, wie bei einer Schrittzähler-App. \parencite{Matusiewicz2017}
	\vspace{\parheadvspace}\\
	\textbf{Wearable Computing}\\
	Ein weiteres Technologiefeld ist wearable Computing. Die bekannteste Anwendung sind hier Smartwatches, welche bei der Sammlung von Daten, wie Schrittanzahl und Puls helfen können, oder im Fall der Pulsmessung, den Patienten bei einem schlechten Wert warnen können. Wearable Computing umfasst hier aber auch strikt medizintechnische Geräte, die immer häufiger Computer enthalten, um die Steuerung und die Auslesung von Daten zu erleichtern. \parencite{Gerke2017}
	\vspace{\parheadvspace}\\
	\textbf{Operations- und Pflege-Roboter}\\
	Schließlich werden in Krankenhäusern Operations- und Pflege-Roboter eingesetzt, um dem Menschen schwere körperliche Arbeit abzunehmen oder Operationen mit einer Präzision durchzuführen, die vorher nicht möglich war. Operations-Roboter versprechen außerdem, dass ein Spezialist eine Operation durchführen kann ohne selbst persönlich anwesend zu sein. Dies kann evtl. kleineren Krankenhäusern helfen, die einen Patienten dann nicht mehr zu einem größeren Krankenhaus überweisen müssen. \parencite{Bardash2010}
	%\parencite[14]{braeutigam2017}
	%\parencite[51]{oswald2019} 
\subsection{IT-Unterstützung der Versorgungsprozesse}
\parencite{oswald2019} \parencite[14]{braeutigam2017}
\begin{itemize}
	\item Krankenhausinformationssystem (KIS)

	\begin{itemize}
		\item Integriert Systeme des Krankenhauses
		\item 
	\end{itemize}
	\item Elektronische Patientenakte (EPA)
	\begin{itemize}
		\item Ein zentraler Ort an dem alle Patientendaten gespeichert werden. Ziel ist es alle benötigten Informationen, wie Stamm- und Falldaten, einfach elektronisch verfügbar zu machen. Um die Akte effizient nutzen zu können ist ein standardisiertes Format wichtig \parencite[62]{oswald2019}.
	\end{itemize}
	\item Patientendatenmanagement (PDMS)
	\begin{itemize}
		\item 
	\end{itemize}
	\item Labor-Informationssysteme (LIS)
	\item Radiologie-Informationssystem (RIS)
	\item Picture Archiving and Communication System (PACS): Bildarchivierungs- und Kommunikationssystem
	\item Cumputerized Physician Order Entry (CPOE): elektronische Arzneimittelverordnung
	\item WLAN
	\item Location Based Services \parencite{mci/Leimeister2006}
\end{itemize}
\subsection{IT-Unterstützung der klinischen Prozesse}
\label{sec:klinischeProzesse}
% <fig>
% \begin{figure}[h]
% 	\centering
% 	\includegraphics[0.8\textwidth]{Bilder/klinische_Kernprozesse_Hübner_2018_S41.png}		
% 	\caption{Die klinischen Kernprozesse \parencite[41]{huebner2018}}
% 	\label{fig:kernprozesse}
% \end{figure}
	\begin{itemize}
		\item Aufnahme, Visite, [OP-Vorbereitung, OP-Nachbereitung], Entlassung (ggf. als Grafik) \parencite[37]{huebner2019}
		\item Im Prozess \textit{Aufnahme} wird ein Patient in das Krankenhaus aufgenommen. Dabei ist es wichtig die Stammdaten und evtl. auch Falldaten zu übernehmen. Für bestmögliche IT-Unterstützung sollen diese direkt digital übernommen werden. 
		\item \textit{Visite} ist der Besuch eines Patienten auf einem Zimmer durch einen Arzt. Hier wird der Begriff etwas ausgedehnt und beinhaltet alle Interaktionen des klinischen Personals mit den Patienten auf Station. Im Gegensatz zur Aufnahme, wo vermutet werden kann, dass datenverarbeitendes Personal Zugang zu einem PC hat, ist es bei der Visite hilfreich, wenn mobile Datenzugänge, wie Notebooks, Tablets oder bedside Terminals, zur Verfügung stehen. Um diese Geräte sinnvoll nutzen zu können ist ein ausreichend ausgebautes krankenhausinternes WLAN nötig.
		\item \textit{OP-Vorbereitung} beschreibt die Planung einer Operation (OP). Das beinhaltet Termin-, Personal- und Raumplanung aber auch die Informationslogistik, dass die für die OP nötigen Daten dem Personal zur richtigen Zeit zur Verfügung stehen. Dies kann beschleunigt werden, wenn die Daten schon in elektronischer Form vorliegen und durch ein integriertes Krankenhausinformationssystem einfach entnommen werden können.
		\item Ähnlich wie in der OP-Vor- geht es in der \textit{OP-Nachbereitung} um die Schnittstelle zwischen OP und anderen Prozessen. Daten die während einer OP generiert wurden sollen an nachstehende Prozesse weitergeleitet werden. Dies so weit wie möglich zu automatisieren ist Aufgabe der IT, indem diese Daten z.B. direkt in eine elektronische Fallakte übernommen werden.
		\item Der Prozess \textit{Entlassung} beschreibt das Ende der stationären Behandlung eines Patienten. Hier müssen evtl. Patientendaten, die während des Aufenthalts entstanden sind, an niedergelassene Ärzte, in Form eines Arztbriefes, weitergeleitet werden. 
		\begin{itemize}
			\item Kommunikation mit Patienten
			\item Erstellung eines Medikationsplans für Patienten
			\item Kommunikation mit ambulanten Gesundheitsdienstleistern
			\item Digitales Diktat ? Was ist ein Diktat?
		\end{itemize}
	\end{itemize}
	\begin{itemize}
	\item In den Kapiteln \ref{sec:stand in deutschland} und \ref{sec:laendervergleich} wird die Sicht der IT auf die klinischen Prozesse für die Analyse in Deutschland und den Ländervergleich mit Österreich und der Schweiz herangezogen.
	\end{itemize}
