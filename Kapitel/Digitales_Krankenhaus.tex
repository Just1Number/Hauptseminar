% !TEX root = ../main.tex
Im digitalen Krankenhaus ist die Verwendung von Papier als Informationsmedium abgeschafft. Um diesen Standard zu erreichen, benötigt es einer Digitalisierung der klinischen Prozesse. Dabei bestehen viele Chancen und Risiken. Um die Chancen zu nutzen und Risiken zu minimieren, gibt es viele verschiedene Technologien, die durch \cite{braeutigam2017} in zwei Bereiche eingeteilt werden: Auf der einen Seite Technologien, die direkt die Behandlung von Patienten betrifft und auf der anderen Seite IT-Funktionen, die die Versorgungsprozesse unterstützen.
\subsection{Chancen und Risiken}
Digitalisierung birgt viele Chancen zur Verbesserung der Abläufe in einem Krankenhaus. Diese reichen von der Organisation bis hin zu medizinisch-fachlichen Innovationen (Siehe Tabelle \ref{tab:hoffnung_risiken}). So wird eine Erleichterung der Dokumentation und eine Verbesserung der Kommunikation versprochen. Auch soll das Personal bei körperlich anstrengenden Tätigkeiten und Routineaufgaben entlastet werden. Die dadurch gewonnene Zeit wird dann bestenfalls für den direkten Patientenkontakt genutzt. Durch neue Technologien wie Operationsroboter kann die Diagnostik und Behandlung unmittelbar verbessert oder in manchen Fällen erst möglich gemacht werden. Wenn Patienten darüber hinaus aus der Ferne betreut werden, können sie sich den Weg zum Krankenhaus komplett sparen. \parencite{braeutigam2017}

Gleichzeitig gibt es wie bei jedem Fortschritt Risiken, die es zu beachten gilt. So kann eine zunehmend digitale Datenlagerung die Datensicherheit reduzieren. Grade im medizinischen Bereich ist diese von enormer Wichtigkeit. Auf der Seite der Arbeitnehmer besteht die Befürchtung, dass eventuelle Zeitersparnis nicht dem Patienten zu gute kommt, sondern zu einer Reduzierung von Arbeitsplätzen führt. Die neuen Aufgaben, die das Personal bewältigen muss, sind komplex und erfordern neue digitale und auch fachliche Fähigkeiten und Kenntnisse. Intensivere Dokumentation birgt das Risko von Überwachung und kann damit den Leistungsdruck steigern. Neue medizintechnische Technologien und damit verbundene neue Systeme sind störanfällig. \parencite{braeutigam2017}
\begin{table}[h]
	\begin{tabular}{l|p{.4\textwidth}|p{.4\textwidth}}
	\textbf{Bereiche}&\textbf{Chancen}&\textbf{Risiken}\\
	\hline
	\multirow{3}{*}{Organisation}
		&$\bullet$ die Erleichterung der Dokumentation &$\bullet$ mangelnde Datensicherheit\\
		&$\bullet$ Verbesserung bei Organisation und Kommunikation &$\bullet$ fehlende Akzeptanz der Beschäftigten\\
	\hline
	\multirow{7}{*}{Personal}
		&$\bullet$ Entlastung von körperlich anstrengenden Tätigkeiten &$\bullet$ wachsender Leistungsdruck und zusätzliche Aufgaben\\
		&$\bullet$ Entlastung von Routineaufgaben &$\bullet$ Überforderung der Beschäftigten\\
		&$\bullet$ Zeitersparnis&$\bullet$ zunehmende Kontrolle und Fremdbestimmung bei der Arbeit\\
		&&$\bullet$ Substitution von Tätigkeiten und Arbeitsplatzverluste\\
	\hline
	\multirow{6}{*}{Fachlich}
		&$\bullet$ Fernbetreuung von Patienten&$\bullet$ Störanfälligkeit\\
		&$\bullet$ die Verbesserung der Versorgungsqualität &$\bullet$ mangelnde Kompetenz bei Beschäftigten\\
		&$\bullet$ Qualitätssteigerungen bei Diagnostik und Therapie, insbesondere bei Operationen&\\
		&$\bullet$ mehr Zeit für den direkten Patientenkontakt vor allem in der Pflege&\\
	\end{tabular}
	\caption{Zusammenstellung der Hoffnungen und Risiken der Digitalisierung an Krankenhäusern \parencite{braeutigam2017}}
	\label{tab:hoffnung_risiken}
\end{table}
\subsection{Behandlungstechnologien}
	Das Kerngeschäft eines Krankenhauses ist die Behandlung von Patienten. Es gibt einige Technologien, mit denen der Patient direkt in Kontakt kommt.
	\cite{braeutigam2017} nennt hier Telemonitoring, mHealth, Wearable Computing sowie Operations- und Pflege-Roboter.
	\vspace{\parheadvspace}\\
	\textbf{Telemonitoring}\\
	Durch Telemonitoring kann sich ein Patient den Weg zum Krankenhaus sparen, indem er bestimmte Messungen selbst durchführt. So kann z.B. ein Blutdruckpatient seinen Blutdruck selbst messen und über eine IT-Anwendung digital an den Arzt übermitteln. 
	\vspace{\parheadvspace}\\
	\textbf{mHealth}\\
	Eng mit Telemonitoring verbunden ist mHealth. mHealth umfasst Anwendungen auf mobilen Geräten, wie Smartphones, die bei der Behandlung unterstützen. Dabei geht es um das übermitteln von Daten, wie etwa Blutdruck, an einen Arzt aber auch das direkte Messen von gesundheitsrelevanten Daten, wie bei einer Schrittzähler-App. \parencite{Matusiewicz2017}
	\vspace{\parheadvspace}\\
	\textbf{Wearable Computing}\\
	Ein weiteres Technologiefeld ist Wearable Computing. Die bekannteste Anwendung sind hier Smartwatches, welche bei der Sammlung von Daten, wie Schrittanzahl und Puls helfen können, oder im Fall der Pulsmessung, den Patienten bei einem schlechten Wert warnen können. Wearable Computing umfasst hier aber auch strikt medizintechnische Geräte, die immer häufiger Computer enthalten, um die Steuerung und die Auslesung von Daten zu erleichtern. \parencite{Gerke2017}
	\vspace{\parheadvspace}\\
	\textbf{Operations- und Pflege-Roboter}\\
	Schließlich werden in Krankenhäusern Operations- und Pflege-Roboter eingesetzt, um den Krankenhausangestellten schwere körperliche Arbeit abzunehmen oder Operationen mit einer Präzision durchzuführen, die vorher nicht möglich war. Operations-Roboter versprechen außerdem, dass ein Spezialist eine Operation durchführen kann, ohne selbst persönlich anwesend zu sein. Dies kann z.B. kleineren Krankenhäusern helfen, die einen Patienten dann nicht mehr in ein größeren Krankenhaus überweisen müssen. \parencite{Bardash2010}
	%\parencite[14]{braeutigam2017}
	%\parencite[51]{oswald2019}
\subsection{IT-Unterstützung der Versorgungsprozesse}
Die Steuerung von Support- und Managementprozessen in der Versorgung ist ein Bereich, in dem bereits intensiv IT-Anwendungen eingesetzt werden. Dabei gibt es für einzelne Abteilungen eigenständige Lösungen, die spezielle Probleme des Feldes lösen. Damit hier abteilungsübergreifend gearbeitet werden kann ist ein Krankenhausinformationssystem (KIS) nötig, dass die Daten dieser Untersysteme zentral zur Verfügung stellt. Das KIS regelt auch, welchen Mitarbeitergruppen welche Zugriffsrechte auf die Daten gestatten werden.\\

Informationen von Patienten werden in einer elektronischen Patientenakte (ePA) verwaltet. Dies ist ein zentraler digitaler Ort, an dem alle Patientendaten gespeichert werden. Ziel ist es, alle benötigten Informationen wie Stamm- und Falldaten, einfach elektronisch verfügbar zu machen. Um die Akte effizient nutzen zu können, ist ein standardisiertes Format wichtig \parencite{oswald2019}. Die ePA ist Teil eines Patientendatenmanagementsystems (PDMS).\\

Spezialisierte Abteilungen haben ihre eigenen Informationssysteme, wie das Labor-Informationssystem (LIS) und das Radiologie-Informationssystem (RIS), die auf die Bedürfnisse der jeweiligen Abteilung zugeschnitten sind.\\

Im Krankenhaus entstehen viele Daten in Form von Bildern. Diese müssen unter Umständen für längere Zeit gespeichert werden, für das ein Picture Archiving and Communication System (PACS) verwendet werden kann.\\

Weitere IT-Lösungen sind die elektronische Arzneimittelverordnung (Cumputerized Physician Order Entry -- CPOE), die zusätzlich durch ein System zur Erstellung eines Medikationsplans unterstützt werden kann, und der elektronische Arztbrief, der die Kommunikation mit niedergelassenen Ärzten vereinfachen soll. \parencite{braeutigam2017}\\

Zusätzlich bietet eine robuste IT-Infrastruktur einige neue Möglichkeiten: So können Daten bei ausgebautem WLAN auch auf mobilen Geräten und somit an dem Ort an dem sie gebraucht werden, aufgerufen werden. Außerdem kann mit Hilfe von Location Based Services ein ortsunabhängiges Medizingerätemanagement verwirklicht werden \parencite{mci/Leimeister2006}.
% \begin{itemize}
% 	\item Krankenhausinformationssystem (KIS)

% 	\begin{itemize}
% 		\item Integriert Systeme des Krankenhauses
% 	\end{itemize}
% 	\item Elektronische Patientenakte (EPA)
% 	\begin{itemize}
% 		\item 
% 	\end{itemize}
% 	\item Patientendatenmanagement (PDMS)
% 	\begin{itemize}
% 		\item 
% 	\end{itemize}
% 	\item Labor-Informationssysteme (LIS)
% 	\item Radiologie-Informationssystem (RIS)
% 	\item Picture Archiving and Communication System (PACS): Bildarchivierungs- und Kommunikationssystem
% 	\item Cumputerized Physician Order Entry (CPOE): elektronische Arzneimittelverordnung
% 	\item WLAN
% 	\item Location Based Services \parencite{mci/Leimeister2006}
% \end{itemize}
\subsection{IT-Unterstützung der klinischen Prozesse}
\label{sec:klinischeProzesse}
	Ein anderer Blickwinkel auf die IT-Funktionen im Krankenhaus ergibt sich mit Blick auf die klinischen Prozesse (Abbildung \ref{fig:kernprozesse}). Der IT-Report Gesundheitswesen \parencite{huebner2019} nennt hier fünf Kernprozesse: Aufnahme, Visite, OP-Vor- und Nachbereitung und Entlassung.
% <fig>
 \begin{figure}[ht]
 	\centering
 	\includegraphics[width=0.8\textwidth]{Bilder/klinische_Kernprozesse_Hübner_2018_S41.png}		
 	\caption{Die fünf klinischen Kernprozesse \parencite[41]{huebner2018}}
 	\label{fig:kernprozesse}
 \end{figure}
	\vspace{\parheadvspace}\\
	\textbf{Aufnahme}\\
		Im Prozess Aufnahme wird ein Patient in das Krankenhaus aufgenommen. Dabei ist es wichtig, die Stammdaten und evtl. auch Falldaten zu übernehmen. Für bestmögliche IT-Unterstützung sollen diese direkt digital übernommen werden.
	\vspace{\parheadvspace}\\
	\textbf{Visite}\\
		Visite ist im engeren Sinne der Besuch eines Patienten auf einem Zimmer durch einen Arzt. Der Begriff wird hier weiter gefasst und beinhaltet alle Interaktionen des klinischen Personals mit den Patienten auf Station. Im Gegensatz zur Aufnahme, wo vermutet werden kann, dass datenverarbeitendes Personal Zugang zu einem PC hat, sind bei der Visite besondere mobile Datenzugänge wie Notebooks, Tablets oder bedside Terminals, hilfreich. Um diese Geräte sinnvoll nutzen zu können, ist ein ausreichend ausgebautes krankenhausinternes WLAN eine wesentliche Voraussetzung.
	\vspace{\parheadvspace}\\
	\textbf{OP-Vorbereitung}\\
		OP-Vorbereitung beschreibt die Planung einer Operation (OP). Das beinhaltet zum einen Termin-, Personal- und Raumplanung, zum anderen aber auch die Informationslogistik, die für die OP nötigen Daten dem Personal zur richtigen Zeit zur Verfügung stellt. Dies kann beschleunigt werden, wenn die Daten schon in elektronischer Form vorliegen und durch ein integriertes Krankenhausinformationssystem einfach entnommen werden können.
	\vspace{\parheadvspace}\\
	\textbf{OP-Nachbereitung}\\
		Ähnlich wie in der OP-Vorbereitung geht es in der OP-Nachbereitung um die Schnittstelle zwischen OP und anderen Prozessen. Daten, die während einer OP generiert wurden, sollen an nachstehende Prozesse weitergeleitet werden. Dies so weit wie möglich zu automatisieren, gelingt, indem diese Daten z.B. direkt in eine elektronische Fallakte übernommen werden.
	\vspace{\parheadvspace}\\
	\textbf{Entlassung}\\
		Der Prozess Entlassung beschreibt das Ende der stationären Behandlung eines Patienten. Hier müssen evtl. Patientendaten, die während des Aufenthalts entstanden sind, an niedergelassene Ärzte weitergeleitet werden. Dies kann in Form eines elektronischen Arztbriefes geschehen.\\

In den Kapiteln \ref{sec:stand in deutschland} und \ref{sec:laendervergleich} wird die Sicht der IT auf die klinischen Prozesse für die Analyse in Deutschland und den Ländervergleich mit Österreich und der Schweiz herangezogen.
