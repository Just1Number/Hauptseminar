% !TEX root = ../main.tex
\subsection{IT-Report Gesundheitswesen 2018}
Der IT-Report Gesundheitswesen wird seit 2002 regelmäßig von der Forschungsgruppe Informatik im Gesundheitswesen (IGW) an der Hochschule Osnabrück veröffentlicht. Es handelt sich dabei um die Vorstellung der Ergebnisse einer Befragung, die sich an leitende Angestellte in Krankenhäusern richtet. Ziel des IT-Report Gesundheitswesen ist es, Informationen über den aktuellen Stand der Nutzung von IT-Systemen zur Verfügung zu stellen und Trends im Zeitverlauf darzustellen. Dabei richtet er sich an ein breites Publikum, von Einrichtungen des Gesundheitswesens und der Wissenschaft bis hin zur IT-Industrie und der Politik. \parencite{huebner2019}\\

Der Stand der Digitalisierung in Deutschland wird hier anhand des IT-Reports Gesundheitswesen 2018 \parencite{huebner2018} und der Analyse des selbigen aus \cite{huebner2019}, vorgestellt. Dem IT-Report Gesundheitswesen 2018 liegt eine Umfrage zugrunde, dessen Feldzeit vom 1.12.2016 bis zum 14.5.2017 lief. Insgesamt wurden 1349 IT-Verantwortliche per E-Mail angeschrieben, die für 1950 Krankenhäuser zuständig waren. Laut statista gab es im Jahr 2016 1951 Krankenhäuser in Deutschland\footnote{https://de.statista.com/statistik/daten/studie/2617/umfrage/anzahl-der-krankenhaeuser-in-deutschland-seit-2000/}, es wurde also versucht alle Krankenhäuser zu erreichen. 224 der Angeschriebenen haben teilgenommen, dies entspricht einer Rücklaufquote von $18,3\%$. Die Erhebung bestand aus 101 Fragen, die in Fragen über die IT-Unterstützung der klinischen Prozesse (vgl. Kapitel \ref{sec:klinischeProzesse}), den Umsetzungsgrad der elektronischen Patientenakte, das IT-Management und allgemeine IT-Funktionen eingeteilt wurden.\\

Zusätzlich wurde für jede Antwort ein Workflow Composite Score (WCS), Professionalism of Information Management Composite Score (PIMCS) und Innovation Capability Composite Score (ICCS) ermittelt. Diese stehen aufbereitet als Statistik zur Verfügung.
\subsection{Defizitbereiche in Deutschland}
\label{sec:defizit_deutschland}
\begin{figure}[ht]
	\centering
 	\includegraphics[width=0.4\textwidth]{Bilder/stand_de_WCS_Hübner_2019.png}
	\caption{Workflow Composite Scores und Sub-Scores der deutschen Krankenhäuser 2017 \parencite[39]{huebner2019}}
	\label{fig:stand_de_WCS}
\end{figure}
Um einen Überblick über den IT-Report Gesundheitswesen 2018 zu erhalten, werden die drei verschiedenen Composite Scores betrachtet, die auf Grundlage der Studie errechnet wurden. Für jeden der Scores sind Mittelwert, Standardabweichung, kleinster und größter Wert angegeben sowie das selbe für die jeweiligen Sub-Scores.
\vspace{\parheadvspace}\\
\textbf{Workflow Composite Score}\\
Der WCS (siehe Abbildung \ref{fig:stand_de_WCS}) konnte für 205 Krankenhäuser errechnet werden, von denen 36 keinen OP haben. Nach der Methodik des WCS werden für die Deskriptoren-Sub-Scores daher nur die Prozesse Aufnahme, Visite und Entlassung betrachtet. Der Mittelwert von 55 bei dem Gesamtscore zeigt ein eher mittelmäßiges Niveau in Deutschland auf. Von den klinischen Prozessen schneidet die Aufnahme am schlechtesten ab (Mittelwert von 44), während die OP Prozesse, vor allem die OP-Vorbereitung, die höchsten Werte (Mittelwert 65) erreichen. Nahe am Gesamtdurchschnitt sind Visite (57) und Entlassung (55). Von den Deskriptoren ist Funktion mit einem Mittelwert von 62 am besten und Distribution mit 52 am schlechtesten. Bei genauerer Betrachtung des Deskriptors Funktion zeigt sich allerdings, dass zwar viele Basisanwendungen wie Dokumentationsfunktionen oder Funktionen für die Leistungsanforderung und Befundrückmeldung eingeführt sind, aber erweiterte Anwendungen wie Funktionen zur Entscheidungsunterstützung meist bestenfalls \glqq in Planung\grqq{ }sind. Im Bereich Daten und Informationen wird schon viel Digital gearbeitet. Es gibt aber immer noch ganze Datenklassen, wie Checklisten oder auffällige Werte, die selten digital vorliegen. Distribution schneidet am schlechtesten ab. Dies liegt unter anderem an einem Mangel an mobilen Endgeräten. In dem meisten Fällen wird mit einem klassischen PC auf digitale Daten zugegriffen. Neuere flexiblere Geräte, wie Tablets, Smartphones/PDAs oder bedside Terminals, sind rar. \parencite{huebner2019}
\vspace{\parheadvspace}\\
\textbf{Professionalism of Information Management Composite Score}\\
Ähnlich wie beim WCS schneiden deutsche Krankenhäuser beim PIMCS (siehe Abbildung \ref{fig:stand_de_PIMCS_ICCS}) mit einem Durchschnitt von 49 aus 100 mittelmäßig ab. Dabei sind die Defizitbereiche auf der taktischen und besonders auf der strategischen Ebene zu suchen. So wird der potenzielle Nutzen von IT-Funktionen selten untersucht und Evaluation der IT-Landschaft findet selten statt. Das Prinzip der IT-Governance wird an deutschen Krankenhäusern selten angewandt.
\begin{figure}[ht]
	\centering
	\includegraphics[width=0.4\textwidth]{Bilder/stand_de_PIMCS_ICCS_Hübner_2019.png}
	\caption{Mittelwert und Standardabweichung für den PIMCS und ICSS \parencite{huebner2019}}
	\label{fig:stand_de_PIMCS_ICCS}
\end{figure}
\vspace{\parheadvspace}\\
\textbf{Innovation Capability Composite Score}\\
Dieses Bild wird bei der Betrachtung des ICCS verstärkt. Es lässt sich ein Mangel an Intrapreneurship der IT-Leitung und eine wenig innovationsfreudige Unternehmenskultur aufdecken. Ein Grund dafür ist, dass die IT meist keine führende Rolle in den Krankenhäusern einnimmt, sondern lediglich die klinische und wirtschaftliche Leitung der Krankenhäuser berät. Zusätzlich sind neue IT-Entwicklungen mit einer niedrigen Priorität innerhalb der ohnehin knappen Finanzierung versehen. Gut schneidet hier die Offenheit gegenüber Anwendern ab: IT-Abteilungen sind also größtenteils gut in das Gesamtunternehmen integriert.

\subsection{Treibergrößen für Digitalisierung in Deutschland}
Der IT-Report Gesundheitswesen 2018 enthält zusätzlich Information über die Demographie der befragten Krankenhäuser. So wird ermittelt, ob das entsprechende Krankenhaus einem Verbund angeschlossen ist, einen Lehrstatus hat, einen OP hat und die Art der Trägerschaft. Damit kann der Unterschied des Mittelwertes des WCS zwischen verschiedenen Krankenhaustypen errechnet werden. Das ergab, dass Krankenhäuser mit OP sowie Universitätskrankenhäuser im Schnitt einen signifikant höheren WCS haben. Auch der Verbundsstatus und die Größe des Krankenhauses, gemessen anhand der stationären Fälle, hat einen positiven Effekt. \parencite{huebner2019}
Ein signifikanter Unterschied zwischen privaten und öffentlichern Trägerschaft kann hingegen nicht festgestellt werden. Andere Studien \parencite{cresswell2013} zeigen wiederrum, dass private Trägerschaft der Digitalisierung förderlich ist.\\

Weiterhin werden von \cite{huebner2019} einige Merkmale stark digitalisierter Krankenhäuser ermittelt.
\vspace{\parheadvspace}\\
\textbf{Innovationskraft}\\
Wenn in einem Krankenhaus die Innovationskraft als ausgeprägt eingeschätzt wird, dann führt das oft zu vermehrtem Einsatz von IT. 
\vspace{\parheadvspace}\\
\textbf{Zusammenarbeit der klinischen Anwender und der IT-Mitarbeiter}\\
Krankenhäuser in denen Kliniker dazu motiviert werden, sich bei IT einzubringen, können diese meist leichter implementieren. Daher ist es wichtig, das Feedback der Anwender zu nutzen und die IT-Strategie und -Taktik entsprechend anzupassen.
\vspace{\parheadvspace}\\
\textbf{Vertrauen und gute Kommunikation zwischen IT und Krankenhausleitung}\\
Nicht nur das Verhältnis von IT zu den Anwendern, sondern vor allem zur Krankenhausleitung ist wichtig für den IT-Fortschritt. Da die Entscheider in Sachen IT meist keine Fachexperten sind müssen sie darauf vertrauen, dass die IT-Abteilung hier gut berät.
\vspace{\parheadvspace}\\
\textbf{Interne und externe Partnerschaften}\\
IT-Systeme in Krankenhäusern sind meist auf die jeweilige Klinik zugeschnitten. Um die IT-Anwendungen dauerhaft gut verwalten zu können, ist eine externe Partnerschaft mit dem IT-Hersteller hilfreich. Auf interner Ebene ist das Vorhandensein einer zentralen IT-Abteilung förderlich.
\vspace{\parheadvspace}\\
\textbf{Eigenverantwortlichkeit}\\
Eine IT-Leitung, die Innovation, Entwicklung und Umsetzung selbst in die Hand nimmt, d.h. motiviert ist, die Digitalisierung im Krankenhaus voranzubringen, kann als weiteres Merkmal hervorgehoben werden.
