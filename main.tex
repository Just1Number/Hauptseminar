% !TEX output_directory=output
%%%%%%%%%%%%%%%%%%%%%%%%%%%%%%%%%%%%%%%%%%%%%%%%%%%%%%%%%%%%%%%%%%%%%%%%%%%%
% !TEX root = main.tex
\documentclass[a4paper,10pt]{article}
\pdfinfo{
     /Title Internationaler Vergleich der Digitalisierung der Krankenhauslandschaft
     /Subject Hauptseminar
     /Author Johannes Wolf
}
\title{Internationaler Vergleich der Digitalisierung der Krankenhauslandschaft:
Treibergrößen und Defizitbereiche in Deutschland}
\author{Johannes Wolf}
\date{\today}

%%%%%%%%%%%%%%%%%%%%%%%%%%%%%%%%%%%%%%%%%%%%%%%%%%%%%%%%%%%%%%%%%%%%%%%%%%%%
% Includes
%%%%%%%%%%%%%%%%%%%%%%%%%%%%%%%%%%%%%%%%%%%%%%%%%%%%%%%%%%%%%%%%%%%%%%%%%%%%
\usepackage[utf8]{inputenc}
\usepackage{csquotes}
\usepackage[german]{babel}
\usepackage[T1]{fontenc}
\usepackage[margin=23mm,bottom=30mm]{geometry}
\usepackage{graphicx, xcolor, wrapfig, helvet, titlesec, fancyhdr}
\usepackage[germankw,german]{algorithm2e}
\usepackage{tikz}
\usepackage{tikz-qtree}
\usepackage{eurosym}
\usetikzlibrary{trees}
\tikzset{
  treenode/.style = {shape=rectangle, rounded corners,
                     draw, align=center},
  root/.style     = {treenode, font=\Large, bottom color=red!30},
  env/.style      = {treenode, font=\ttfamily\normalsize},
  dummy/.style    = {circle,draw}
}
\usepackage{amsmath,amsfonts,amssymb,amsthm}
\usepackage{hyperref} % verwandelt alle Kapitelüberschriften, Verweise aufs Literaturverzeichnis und andere Querverweise in PDF-Hyperlinks
\hypersetup{
	pdftitle={Internationaler Vergleich der Digitalisierung der Krankenhauslandschaft},
	pdfauthor={Johannes Wolf},
	pdfborder={0 0 0} % entfernt hässliche Kästen um Links
}
\usepackage[%
  backend=bibtex      % biber or bibtex
 ,style=authortitle
 ,citestyle=authoryear-comp %authoryear-icomp
 % ,sorting=none        % no sorting
 ,sorting=nyt         % name, year, title
 ,dashed=false        % so author names aren't replaced by a dash
 ,block=none
 ,indexing=false
 ,citereset=none
 ,isbn=true
 ,url=true
 ,doi=true            % prints doi (digital object identifier)
 ,natbib=true         % if you need natbib functions
 ,language=german
 ,maxcitenames=1			% Cite as 'Name et al.' if there are multiple authors
]{biblatex}
\addbibresource{Quellen.bib}  % better than \bibliography
\DefineBibliographyStrings{german}{
	bibliography = {Literaturverzeichnis},
  andothers = {et al.}
}
\defbibheading{bibliography}[\bibname]{%
	 \noindent\LARGE\textbf{\textcolor{thmgrey}#1}
}

% define color used for titels. Offical color of the thm
\definecolor{thmgrey}{HTML}{4a5c66}

% set font to Arial look-alike helvet
\renewcommand{\familydefault}{\sfdefault}

% Defines for mathematical notation. Add additional defines as needed.
\def\O{\mathcal{O}}

% Set paragraph indentation and spacing
\setlength{\parindent}{0pt}
\setlength{\parskip}{0.5ex}

% All this just to have page numbers not centered smh
% \pagestyle{fancy}
% \fancyhf{}
% \renewcommand{\headrulewidth}{0pt}
% \renewcommand{\footrulewidth}{0pt}
% \rfoot{\thepage}

% Reformat section
\titleformat
{\section} % command
[block] % shape
{\LARGE\bfseries\color{thmgrey}} % format
{\thesection} % label
{1ex} % sep between label and title
{} % before-code
{} % after-code

% Reformat subsection
\titleformat
{\subsection} % command
[block] % shape
{\large\bfseries\color{thmgrey}} % format
{\thesubsection} % label
{1ex} % sep between label and title
{} % before-code
{} % after-code

% Reformat subsubsection
\titleformat
{\subsubsection} % command
[block] % shape
{\large\bfseries\color{thmgrey}} % format
{\thesubsubsection} % label
{1ex} % sep between label and title
{} % before-code
{} % after-code
 % packages and global definitions
%% !TEX root = main.tex
% Definition of the Cover Page
\def\seminarheader{{
  \pagenumbering{Roman}
  \hypersetup{pageanchor=false}
  \begin{titlepage}
    \pagestyle{empty}
    % header (thm logo)
    \begin{wrapfigure}{r}{0.68\textwidth}
      \centering
      \includegraphics[width=0.68\textwidth]{Bilder/Logo_THM_FB13.png}
    \end{wrapfigure}
    \parbox[t]{0.32\textwidth}{
      \vspace{-0.5ex}
      \color{thmgrey}
      Hauptseminar\\*
      Dr. Eschenhof-Kammer \\*
      Wintersemester 2020/21 \\*
    }

    % Title of the Paper
    \vspace*{\fill} % hack hack hackedy hack
    \parbox[t]{0.95\textwidth}{
      \center\LARGE\bf\color{thmgrey} 
      Seminararbeit\\*
      \vspace{1ex}
      Internationaler Vergleich der Digitalisierung der Krankenhauslandschaft:\\
Treibergrößen und Defizitbereiche in Deutschland\\*
    }
    \vspace*{\fill}

    % Contact information
    \parbox[t]{\textwidth}{
      \large Johannes Wolf \\*
      \textcolor{thmgrey}{Email:} johannes.wolf@mnd.thm.de \\*
      \textcolor{thmgrey}{Matrikelnummer:} 5146451 \\*
    }
  \end{titlepage}
  % new page
  \hypersetup{pageanchor=true}
  % \pagenumbering{Roman}
  % \tableofcontents
  % \newpage
}}

% Add header to the beginning of the document
\AtBeginDocument{\seminarheader}
 % titelpage and table of contents
%%%%%%%%%%%%%%%%%%%%%%%%%%%%%%%%%%%%%%%%%%%%%%%%%%%%%%%%%%%%%%%%%%%%%%%%%%%%
\begin{document}
%\section*{Abstract}
%\addcontentsline{toc}{section}{Abstract}
%\newpage
\pagenumbering{arabic}
\section{Einleitung}
	\subsection{Was beinhaltet Digitalisierung?}
	\parencite[11]{braeutigam2017}
\section{Das digitale Krankenhaus}
	Erster Satz. Im Krankenhaus gibt es enormes Potential für digitale Anwendungen. 

	Krankenhaus ähnlich wie hochspezialisiertes produzierendes Unternehmen

	Es gibt verschiedene Ansätze die im Krankenhaus angewandten Technologien einzuteilen

	Chancen / Risiken (evtl zusätzlich Tabelle zum gegenüberstellen)
	\subsection{Treibergrößen}
	\begin{itemize}
		\item Hoffnungen \parencite[15]{braeutigam2017}
		\begin{itemize}
			\item die Verbesserung der Versorgungsqualität
			\item die Erleichterung der Dokumentation
			\item Verbesserung bei Organisation und Kommunikation
			\item Qualitätssteigerungen bei Diagnostik und Therapie, insbesondere bei Operationen
			\item Zeitersparnis
			\item mehr Zeit für den direkten Patientenkontakt vor allem in der Pflege
			\item Entlastung von Routineaufgaben
			\item Entlastung von körperlich anstrengenden Tätigkeiten
			\item Fernbetreuung von Patienten
		\end{itemize}
	\end{itemize}
	\begin{itemize}
		\item Beispiele
		\begin{itemize}
			\item "Beispielweise wird Software angeboten, die aufgrund eingespeister Informationen zu einzelnen Patienten Pflegemaßnahmen vorgibt. Im ärztlichen Bereich wird die Standardisierung, Automatisierung und Kontrolle bei chirurgischen Eingriffen thematisiert." \parencite[16]{braeutigam2017}
			\item 
		\end{itemize}
	\end{itemize}
	\subsection{Behandlungstechnologien}
		Das Kerngeschäft eines Krankenhauses ist die Behandlung von Patienten. Es gibt einige Technologien, mit denen der Patient direkt in Kontakt kommt.
		\cite{braeutigam2017} nennt hier Telemonitoring, mHealth, Wearable Computing, Operations- und Pflege-Roboter.
		\vspace{1ex}\\
		\textbf{Telemonitoring}\\
		Durch Telemonitoring kann sich ein Patient den Weg zum Krankenhaus sparen kann, indem er bestimmte Messungen selbst durchführt. So kann z.B. ein Blutdruckpatient seinen eigenen Blutdruck messen und über eine IT-Anwendung digital an den Arzt übermitteln. 
		\vspace{1ex}\\
		\textbf{mHealth}\\
		Eng mit Telemonitoring verbunden ist mHealth. mHealth umfasst Anwendungen auf mobilen Geräten, wie Smartphones, die bei der Behandlung unterstützen. Dabei geht es um das übermitteln von Daten, wie etwa Blutdruck, an einen Arzt aber auch das direkte Messen von gesundheitsrelevanten Daten, wie bei einer Schrittzähler-App. \parencite{Matusiewicz2017}
		\vspace{1ex}\\
		\textbf{Wearable Computing}\\
		Ein weiteres Technologiefeld ist wearable Computing. Die bekannteste Anwendung sind hier Smartwatches, welche bei der Sammlung von Daten, wie Schrittanzahl und Puls helfen können, oder im Fall der Pulsmessung, den Patienten bei einem schlechten Wert warnen können. Wearable Computing umfasst hier aber auch strikt medizintechnische Geräte, die immer häufiger Computer enthalten, um die Steuerung und die Auslesung von Daten zu erleichtern. \parencite{Gerke2017}
		\vspace{1ex}\\
		\textbf{Operations- und Pflege-Roboter}\\
		Schließlich werden in Krankenhäusern Operations- und Pflege-Roboter eingesetzt, um dem Menschen schwere körperliche Arbeit abzunehmen oder Operationen mit einer Präzision durchzuführen, die vorher nicht möglich war. Operations-Roboter versprechen außerdem, dass ein Spezialist eine Operation durchführen kann ohne selbst persönlich anwesend zu sein. Dies kann evtl. kleineren Krankenhäusern helfen, die einen Patienten dann nicht mehr zu einem größeren Krankenhaus überweisen müssen. \parencite{Bardash2010}
		%\parencite[14]{braeutigam2017}
		%\parencite[51]{oswald2019} 
	\subsection{IT-Unterstützung der Versorgungsprozesse}
	\parencite{oswald2019} \parencite[14]{braeutigam2017}
	\begin{itemize}
		\item Krankenhausinformationssystem (KIS)

		\begin{itemize}
			\item Integriert Systeme des Krankenhauses
			\item 
		\end{itemize}
		\item Elektronische Patientenakte (EPA)
		\begin{itemize}
			\item Ein zentraler Ort an dem alle Patientendaten gespeichert werden. Ziel ist es alle benötigten Informationen, wie Stamm- und Falldaten, einfach elektronisch verfügbar zu machen. Um die Akte effizient nutzen zu können ist ein standardisiertes Format wichtig \parencite[62]{oswald2019}.
		\end{itemize}
		\item Patientendatenmanagement (PDMS)
		\begin{itemize}
			\item 
		\end{itemize}
		\item Labor-Informationssysteme (LIS)
		\item Radiologie-Informationssystem (RIS)
		\item Picture Archiving and Communication System (PACS): Bildarchivierungs- und Kommunikationssystem
		\item Cumputerized Physician Order Entry (CPOE): elektronische Arzneimittelverordnung
		\item WLAN
		\item Location Based Services \parencite{mci/Leimeister2006}
	\end{itemize}
	\subsection{IT-Unterstützung der klinischen Prozesse}
	% <fig>
	% \begin{figure}[h]
	% 	\centering
	% 	\includegraphics[0.8\textwidth]{Bilder/klinische_Kernprozesse_Hübner_2018_S41.png}		
	% 	\caption{Die klinischen Kernprozesse \parencite[41]{huebner2018}}
	% 	\label{fig:kernprozesse}
	% \end{figure}
		\begin{itemize}
			\item Klinische Prozesse
			\begin{itemize}
				\item Aufnahme, Visite, [OP-Vorbereitung, OP-Nachbereitung], Entlassung (ggf. als Grafik) \parencite[37]{huebner2019}
				\item Im Prozess \textit{Aufnahme} wird ein Patient in das Krankenhaus aufgenommen. Dabei ist es wichtig die Stammdaten und evtl. auch Falldaten zu übernehmen. Für bestmögliche IT-Unterstützung sollen diese direkt digital übernommen werden. 
				\item \textit{Visite} ist der Besuch eines Patienten auf einem Zimmer durch einen Arzt. Hier wird der Begriff etwas ausgedehnt und beinhaltet alle Interaktionen des klinischen Personals mit den Patienten auf Station. Im Gegensatz zur Aufnahme, wo vermutet werden kann, dass datenverarbeitendes Personal Zugang zu einem PC hat, ist es bei der Visite hilfreich, wenn mobile Datenzugänge, wie Notebooks, Tablets oder bedside Terminals, zur Verfügung stehen. Um diese Geräte sinnvoll nutzen zu können ist ein ausreichend ausgebautes krankenhausinternes WLAN nötig.
				\item \textit{OP-Vorbereitung} beschreibt die Planung einer Operation (OP). Das beinhaltet Termin-, Personal- und Raumplanung aber auch die Informationslogistik, dass die für die OP nötigen Daten dem Personal zur richtigen Zeit zur Verfügung stehen. Dies kann beschleunigt werden, wenn die Daten schon in elektronischer Form vorliegen und durch ein integriertes Krankenhausinformationssystem einfach entnommen werden können.
				\item Ähnlich wie in der OP-Vor- geht es in der \textit{OP-Nachbereitung} um die Schnittstelle zwischen OP und anderen Prozessen. Daten die während einer OP generiert wurden sollen an nachstehende Prozesse weitergeleitet werden. Dies so weit wie möglich zu automatisieren ist Aufgabe der IT, indem diese Daten z.B. direkt in eine elektronische Fallakte übernommen werden.
				\item Der Prozess \textit{Entlassung} beschreibt das Ende der stationären Behandlung eines Patienten. Hier müssen evtl. Patientendaten, die während des Aufenthalts entstanden sind, an niedergelassene Ärzte, in Form eines Arztbriefes, weitergeleitet werden. 
				\begin{itemize}
					\item Kommunikation mit Patienten
					\item Erstellung eines Medikationsplans für Patienten
					\item Kommunikation mit ambulanten Gesundheitsdienstleistern
					\item Digitales Diktat ? Was ist ein Diktat?
				\end{itemize}
			\end{itemize}
			\item Deskriptoren
		\begin{itemize} 
			\item Information und Daten
			\item IT-Funktionen/Anwendungen
			\item Integration
			\item Distribution
		\end{itemize}
		\item In den Kapiteln \ref{sec:stand in deutschland} und \ref{sec:laendervergleich} wird die Sicht der IT auf die klinischen Prozesse für die Analyse in Deutschland und den Ländervergleich mit Österreich und der Schweiz herangezogen.
		\end{itemize}
	\subsection{IT-Report}
	IT-Report der Uni Osnabrück beschreiben
\section{Methoden zur Messung des Digitalisierungsstandes}
	\subsection{Workflow Composite Score}
		Aufnahme, Visite, [Op-Vorbereitung, Op-Nachbereitung], Entlassung (ggf. als Grafik)\\
		% <fig>
		%\begin{figure}[h]
		%	\centering
		%	\includegraphics[width=0.8\textwidth]{Bilder/WCS_Hübner_2018_S22.png}
		%	\caption{Hierarchie des Workflow Composite Scores \parencite[22]{huebner2018}}
		%	\label{fig:WCS}
		%\end{figure}
		Diese Kennzahl bedient sich des \textit{Prinzip der klinischen Informationslogistik}. Bei diesem Prinzip \glqq{}sollte ein [Krankenhausinformationssystem] immer die richtigen Patienteninformation zur richtigen Zeit am richtigen Ort in der richtigen Qualität für die richtigen Personen bereitstellen.\grqq\parencite[36]{huebner2019} Der Workflow Composite Score misst inwieweit dieser Standard erreicht ist. Hierbei werden über 100 Einzelindikatoren erfasst und, wenn möglich, in 5 klinische Prozesse (Aufnahme, Visite, OP-Vorbereitung, OP-Nachbereitung, Entlassung) eingeteilt. Bei Krankenhäusern, die nicht über einen OP verfügen, werden nur Aufnahme, Visite und Entlassung betrachtet. Innerhalb dieser Prozesse werden die Indikatoren noch 4 Deskriptoren aufgeteilt. Diese Indikatoren sind
		\begin{enumerate}
			\item Daten und Informationen
			\item Funktionen
			\item Integration und
			\item Distribution.
		\end{enumerate}
		Bei \textit{Daten und Informationen} geht es um die Vielfalt der elektronisch verfügbaren Patienteninformationen. Teil dieses Deskriptors ist z.B. die Frage ob Patientenstammdaten bei der Aufnahme schon in elektronischer Form vorhanden sind oder ob sie erst manuell aufgenommen werden müssen. \textit{Funktion} erfasst welche IT-Anwendungen zur Verfügung stehen, z.B. ein System zur Belegesteuerung, aber auch das Vorhandensein von WLAN. Inwieweit diese Anwendungen miteinander verknüpft sind zeigt der Deskriptor \textit{Integration} auf. \textit{Distribution} sind die Mittel mit denen Informationen an ihre Nutzer gebracht werden. Können Pflegekräfte Patientendaten nur an einem PC einsehen oder gibt es Tablets, die einfacher mit auf ein Zimmer genommen werden können?

		Für die Schnittstelle
	\subsection{ICS}
	\subsection{PIMCS}
	\subsection{EMRAM}
	\parencite{Stephani2019}
\section{Aktueller Stand der Digitalisierung in deutschen Krankenhäuser} \label{sec:stand in deutschland}
	\subsection{Aktueller Stand bzgl. Bewertungskriterien}
		Veränderung von 2018 zu 2020
	\subsection{Treibergrößen für Digitalisierung in Deutschland}
	\begin{itemize}
		\item Methodik
		\begin{itemize}
			\item Betrachtung des WCS über KH-Attribute (z.B. Trägerschaft)
			\item (mind. 0,05 Signifikanz des Mittelwertunterschieds)
		\end{itemize}
	\end{itemize}
	\begin{itemize}
		\item Verbundstatus \parencite{cresswell2013}
		\item Lehrstatus
		\item OP
		\item Größe \parencite{cresswell2013}
		\item nicht: Trägerschaft \parencite{cresswell2013}
	\end{itemize}
	\begin{itemize}
		\item Zusammenarbeite der klinischen Anwender und IT-Mitarbeiter
		\item Vertrauen und gute Kommunikation zwischen IT und Krankenhausleitung
		\item Interne und externe Partnerschaften
		\begin{itemize}
			\item Zusammenarbeit mit IT-Herstellern
			\item zentrale IT-Abteilung
		\end{itemize}
		\item Eigenverantwortlichkeit der einzelnen Abteilungen
		\item Innovationskraft
	\end{itemize}
	\subsection{Defizite / Defizitbereiche}
	\begin{itemize}
		\item IT-Governance
		\item Distribution von Daten (Endgeräte)
		\item Politik-Gedöns
	\end{itemize}
\section{Ländervergleich} \label{sec:laendervergleich}
% !TEX root = main.tex
\begin{itemize}
	\item Gegenüberstellung der Fragen im IT-Report 2020 \parencite{huebner2020}
	\item Wie viele Krankenhäuser wurden befragt?
	\begin{itemize}
		\item 608 IT-Verantwortliche
		\item 492 Deutschland
		\item 49 Österreich
		\item 67 Schweiz
		\item Rücklaufquoten:
		\begin{itemize}
			\item 25,2\% (DE), 18,8\% (AT), 31,8\% (CH)
		\end{itemize}
	\end{itemize}
	\item Wie viele Krankenhäuser gibt es insgesamt
	\item Gemeinsamkeiten und Unterschiede aufzeigen
\end{itemize}
\subsection{Gegenüberstellung der Länder}
\subsection{Defizite / Defizitbereiche im Vergleich}
\section{Fazit und Ausblick}
\begin{itemize}
	\item Treibergröße Geld
\end{itemize}
\newpage
\pagenumbering{Roman}
\setcounter{page}{3}
%\printbibliography
%\addcontentsline{toc}{section}{Literaturverzeichnis}
\end{document}
