% !TEX output_directory=output
%%%%%%%%%%%%%%%%%%%%%%%%%%%%%%%%%%%%%%%%%%%%%%%%%%%%%%%%%%%%%%%%%%%%%%%%%%%%
% !TEX root = main.tex
\documentclass[a4paper,10pt]{article}
\pdfinfo{
     /Title Internationaler Vergleich der Digitalisierung der Krankenhauslandschaft
     /Subject Hauptseminar
     /Author Johannes Wolf
}
\title{Internationaler Vergleich der Digitalisierung der Krankenhauslandschaft:
Treibergrößen und Defizitbereiche in Deutschland}
\author{Johannes Wolf}
\date{\today}

%%%%%%%%%%%%%%%%%%%%%%%%%%%%%%%%%%%%%%%%%%%%%%%%%%%%%%%%%%%%%%%%%%%%%%%%%%%%
% Includes
%%%%%%%%%%%%%%%%%%%%%%%%%%%%%%%%%%%%%%%%%%%%%%%%%%%%%%%%%%%%%%%%%%%%%%%%%%%%
\usepackage[utf8]{inputenc}
\usepackage{csquotes}
\usepackage[german]{babel}
\usepackage[T1]{fontenc}
\usepackage[margin=23mm,bottom=30mm]{geometry}
\usepackage{graphicx, xcolor, wrapfig, helvet, titlesec, fancyhdr}
\usepackage[germankw,german]{algorithm2e}
\usepackage{tikz}
\usepackage{tikz-qtree}
\usepackage{eurosym}
\usetikzlibrary{trees}
\tikzset{
  treenode/.style = {shape=rectangle, rounded corners,
                     draw, align=center},
  root/.style     = {treenode, font=\Large, bottom color=red!30},
  env/.style      = {treenode, font=\ttfamily\normalsize},
  dummy/.style    = {circle,draw}
}
\usepackage{amsmath,amsfonts,amssymb,amsthm}
\usepackage{hyperref} % verwandelt alle Kapitelüberschriften, Verweise aufs Literaturverzeichnis und andere Querverweise in PDF-Hyperlinks
\hypersetup{
	pdftitle={Internationaler Vergleich der Digitalisierung der Krankenhauslandschaft},
	pdfauthor={Johannes Wolf},
	pdfborder={0 0 0} % entfernt hässliche Kästen um Links
}
\usepackage[%
  backend=bibtex      % biber or bibtex
 ,style=authortitle
 ,citestyle=authoryear-comp %authoryear-icomp
 % ,sorting=none        % no sorting
 ,sorting=nyt         % name, year, title
 ,dashed=false        % so author names aren't replaced by a dash
 ,block=none
 ,indexing=false
 ,citereset=none
 ,isbn=true
 ,url=true
 ,doi=true            % prints doi (digital object identifier)
 ,natbib=true         % if you need natbib functions
 ,language=german
 ,maxcitenames=1			% Cite as 'Name et al.' if there are multiple authors
]{biblatex}
\addbibresource{Quellen.bib}  % better than \bibliography
\DefineBibliographyStrings{german}{
	bibliography = {Literaturverzeichnis},
  andothers = {et al.}
}
\defbibheading{bibliography}[\bibname]{%
	 \noindent\LARGE\textbf{\textcolor{thmgrey}#1}
}

% define color used for titels. Offical color of the thm
\definecolor{thmgrey}{HTML}{4a5c66}

% set font to Arial look-alike helvet
\renewcommand{\familydefault}{\sfdefault}

% Defines for mathematical notation. Add additional defines as needed.
\def\O{\mathcal{O}}

% Set paragraph indentation and spacing
\setlength{\parindent}{0pt}
\setlength{\parskip}{0.5ex}

% All this just to have page numbers not centered smh
% \pagestyle{fancy}
% \fancyhf{}
% \renewcommand{\headrulewidth}{0pt}
% \renewcommand{\footrulewidth}{0pt}
% \rfoot{\thepage}

% Reformat section
\titleformat
{\section} % command
[block] % shape
{\LARGE\bfseries\color{thmgrey}} % format
{\thesection} % label
{1ex} % sep between label and title
{} % before-code
{} % after-code

% Reformat subsection
\titleformat
{\subsection} % command
[block] % shape
{\large\bfseries\color{thmgrey}} % format
{\thesubsection} % label
{1ex} % sep between label and title
{} % before-code
{} % after-code

% Reformat subsubsection
\titleformat
{\subsubsection} % command
[block] % shape
{\large\bfseries\color{thmgrey}} % format
{\thesubsubsection} % label
{1ex} % sep between label and title
{} % before-code
{} % after-code
 % packages and global definitions
% !TEX root = main.tex
% Definition of the Cover Page
\def\seminarheader{{
  \pagenumbering{Roman}
  \hypersetup{pageanchor=false}
  \begin{titlepage}
    \pagestyle{empty}
    % header (thm logo)
    \begin{wrapfigure}{r}{0.68\textwidth}
      \centering
      \includegraphics[width=0.68\textwidth]{Bilder/Logo_THM_FB13.png}
    \end{wrapfigure}
    \parbox[t]{0.32\textwidth}{
      \vspace{-0.5ex}
      \color{thmgrey}
      Hauptseminar\\*
      Dr. Eschenhof-Kammer \\*
      Wintersemester 2020/21 \\*
    }

    % Title of the Paper
    \vspace*{\fill} % hack hack hackedy hack
    \parbox[t]{0.95\textwidth}{
      \center\LARGE\bf\color{thmgrey} 
      Seminararbeit\\*
      \vspace{1ex}
      Internationaler Vergleich der Digitalisierung der Krankenhauslandschaft:\\
Treibergrößen und Defizitbereiche in Deutschland\\*
    }
    \vspace*{\fill}

    % Contact information
    \parbox[t]{\textwidth}{
      \large Johannes Wolf \\*
      \textcolor{thmgrey}{Email:} johannes.wolf@mnd.thm.de \\*
      \textcolor{thmgrey}{Matrikelnummer:} 5146451 \\*
    }
  \end{titlepage}
  % new page
  \hypersetup{pageanchor=true}
  % \pagenumbering{Roman}
  % \tableofcontents
  % \newpage
}}

% Add header to the beginning of the document
\AtBeginDocument{\seminarheader}
 % titelpage and table of contents
%%%%%%%%%%%%%%%%%%%%%%%%%%%%%%%%%%%%%%%%%%%%%%%%%%%%%%%%%%%%%%%%%%%%%%%%%%%%
\begin{document}
%\section*{Abstract}
%\addcontentsline{toc}{section}{Abstract}
%\newpage
\pagenumbering{arabic}
\section{Einleitung}
\section{Das digitale Krankenhaus}
	\subsection{[Klinische Prozesse]}
		Aufnahme, Visite, [OP-Vorbereitung, OP-Nachbereitung], Entlassung (ggf. als Grafik)
	\subsection{IT-Unterstützung klinischer Prozesse}
		\parencite[37]{huebner2019}
	\begin{itemize} 
		\item Information und Daten
		\item IT-Funktionen/Anwendungen
		\item Integration
		\item Distribution
	\end{itemize}
	\subsection{ Elektronische Patientenakte}
	\subsection{IT-Management?}
\section{Bewertungskriterien und Treibergrößen des Digitalisierungsfortschritts}
	\subsection{Workflow Composite Score}
		Matrix “Klinischer Prozess” x “IT-Unterstützung/Deskriptoren”
	\subsection{ICS}
	\subsection{PIMCS}
	\subsection{Treibergrößen (definieren und gemeinsames Verständnis definieren)}
	\begin{itemize}
		\item Methodik
		\begin{itemize}
			\item Betrachtung des WCS über KH-Attribute (z.B. Trägerschaft)
			\item (mind. 0,05 Signifikanz des Mittelwertunterschieds)
		\end{itemize}
	\end{itemize}
\section{Aktueller Stand der Digitalisierung in deutschen Krankenhäuser}
	\subsection{Aktueller Stand bzgl. Bewertungskriterien}
		Veränderung von 2018 zu 2020
	\subsection{Treibergrößen für Digitalisierung in Deutschland}
	\begin{itemize}
		\item Verbundstatus \parencite{cresswell2013}
		\item Lehrstatus
		\item OP
		\item Größe \parencite{cresswell2013}
		\item nicht: Trägerschaft \parencite{cresswell2013}
	\end{itemize}
	\subsection{Defizite / Defizitbereiche}
	\begin{itemize}
		\item IT-Governance
		\item Distribution von Daten (Endgeräte)
		\item Politik-Gedöns
	\end{itemize}
\section{Ländervergleich}
% !TEX root = main.tex
\begin{itemize}
	\item Gegenüberstellung der Fragen im IT-Report 2020 \parencite{huebner2020}
	\item Wie viele Krankenhäuser wurden befragt?
	\begin{itemize}
		\item 608 IT-Verantwortliche
		\item 492 Deutschland
		\item 49 Österreich
		\item 67 Schweiz
		\item Rücklaufquoten:
		\begin{itemize}
			\item 25,2\% (DE), 18,8\% (AT), 31,8\% (CH)
		\end{itemize}
	\end{itemize}
	\item Wie viele Krankenhäuser gibt es insgesamt
	\item Gemeinsamkeiten und Unterschiede aufzeigen
\end{itemize}
\subsection{Gegenüberstellung der Länder}
\subsection{Defizite / Defizitbereiche im Vergleich}
\section{Fazit und Ausblick}
\newpage
\pagenumbering{Roman}
\setcounter{page}{3}
\printbibliography
\addcontentsline{toc}{section}{Literaturverzeichnis}
\end{document}
